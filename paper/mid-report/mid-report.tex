\documentclass[12pt]{report}
\usepackage[ruled,vlined]{algorithm2e} %Uses version 3.9
\usepackage{url}

\newcommand{\TODO}[1]{}

\begin{document}
\title{Reducing Contention in Concurrent Queues: Midterm Report}
\author{Carlos Valera \\
    University of Central Florida \\
    \texttt{cvaleraleon@knights.ucf.edu}}
\date{March 15, 2013}
\maketitle
\begin{abstract}
I introduce the concepts of operation for a new contention management system
for concurrent queues called a MultiQueue. The MultiQueue wraps concurrent
queue algorithms from previous authors, and reduces contention on accesses to
these queues from multiple threads by distributing data across multiple
structure instances. I plan to test the throughput MultiQueue under various
load conditions and compare the results against previous implementations of
queues without the MultiQueue.  As this is a preliminary report, work on this
structure is ongoing, and the functionality of the MultiQueue has not yet been
finalized.
\end{abstract}
\section{Introduction}
%Problem Description
The concurrent queue is a well-studied and practical structure in concurrent
programming. One issue with the concurrent queue is that because queues are an
inherently sequential data-structure, many implementations of concurrent queues
suffer from high contention when under load. I will describe a method for
managing contention between queues to gain performance. I call the structure
which does this a MultiQueue. The method builds on previous implementations of
concurrent queues, and tries to improve concurrency by distributing the work
presented to the queue amongst multiple instances of concurrent queues.

There still remains a good deal of work to be done on the MultiQueue as of the
time of this writing. A good deal of this paper will go over possible avenues
of exploration to improve the MultiQueue, and on issues which might be faced in
development.

%History
\section{Algorithms}
%Concepts
The concepts surrounding the MultiQueue are simple enough. The structure keeps
an internal list of concurrent queues and two counters that keep track how many
enqueue and dequeue operations have been made on the MultiQueue. In essence,
the internal list is treated like a circular array based queue of queues, and
the two counters essentially point to the head and tail of this queue. The
intent is to isolate the larger enqueue and dequeue operations of more complex
queue algorithms so that, on average, each queue has only one producer and one
consumer operating on it.

%Pseudocode
\begin{scriptsize}
\begin{algorithm}[H]
\SetKwIF{blk}{}{}{}{}{}{}{} % Hacky way to get vline to work properly with blocks
\SetKwFunction{Init}{void initialize}
\SetKwFunction{Enq}{void enqueue}
\SetKwFunction{Deq}{bool dequeue}
\SetKwFunction{ienq}{internalEnqueue}
\SetKwFunction{ideq}{internalDequeue}
\SetKwFunction{iinit}{internalInitialize}
\SetKwFunction{FAA}{fetchAndAdd}
\SetKwFunction{nexttwo}{nextPowerOf2}
\SetKwFunction{ladd}{listAdd}
\SetKw{Continue}{continue}
\SetKw{Struct}{structure}
\SetKwFor{For}{for}{}{}
\SetKwIF{If}{ElseIf}{Else}{if}{}{elif}{else}{}
\SetKwFor{While}{while}{}{}
\SetKwInOut{Input}{input}
\SetKwInOut{Output}{output}
\SetNoFillComment

\caption{Pseudocode defining the operations which can be done on a MultiQueue
         as well as the structure of a MultiQueue.}

\Struct InternalQueue
\blk{}{
    \tcp{The internal structure of this is not important as long as supports
    the concurrent enqueue and dequeue operations}
}

\Struct MultiQueue
\blk{}{
    int num\_queues\;
    int mask \tcp*[r]{Bitmask used to wrap counters}
    uint producer\_counter \tcp*[r]{Tracks the next "empty" spot for a producer}
    uint consumer\_counter \tcp*[r]{Tracks the next "empty" spot for a consumer}
    list queues \tcp*[r]{A list of internal concurrent queues}
}
\BlankLine
\BlankLine
\BlankLine
\BlankLine
\BlankLine

\Input{The multi queue to initialize and the number of queues to initialize it
       for}
\Init{MultiQueue Q, int num\_queues}
\blk{}{
    Q.producer\_counter = Q.consumer\_counter = 0\;
    Q.num\_queues = \nexttwo{num\_queues}\;
    Q.mask = Q.num\_queues$-$1\;
    Q.queues = list()\;
    \For{i=0 \KwTo Q.num\_queues}{
        IQ = InternalQueue\;
        \iinit{IQ}\;
        \ladd{Q.queues, IQ}
    }
}
\BlankLine
\BlankLine
\BlankLine
\BlankLine
\BlankLine

\Input{The multi queue to enqueue into and the item to enqueue}
\Enq{MultiQueue  Q, data\_t\_ptr item}
\blk{}{
    next\_empty = \FAA{Q.producer\_counter, 1}\;
    next\_queue = Q.queues[next\_empty]\;
    \ienq{next\_queue, item}\;
}
\BlankLine
\BlankLine
\BlankLine
\BlankLine
\BlankLine

\Input{The multi queue to attempt to dequeue from and an address to a location
       to store the dequeued item}
\Output{If the queue is empty, false is returned, otherwise true is returned
        and the value of the dequeued item is copied into the address pointed to by
        \it{result}}
\Deq{MultiQueue Q, data\_t\_ptr result}
\blk{}{
    \If(\tcp*[f]{quit if queue is empty}){Q.consumer\_ptr$<$Q.producer\_counter}{
        \Return{false}\;
    }
    next\_empty = \FAA{Q.consumer\_counter, 1}\;
    next\_queue = Q.queues[next\_empty $\&$ Q.mask]\;
    \BlankLine
    \tcp{Keep trying to dequeue an item until one is found. This loop is needed
         to prevent race conditions between a consumer dequeueing and a producer
         enqueueing at the same time}
    \While{$\neg$\ideq{next\_queue, result}}{
        \Continue
    }
    \Return{true}\;
    
}
\end{algorithm}
\end{scriptsize}

\section{Work to be Done}
\subsection{Proof of Correctness}
In this document I do not have a proof of correctness of a MultiQueue or even a
proof for it having any desirable attributes. In it's current state, I suspect
that the MultiQueue is a linearizable structure because it relies on
linearizable implementations of concurrent queues to function correctly.
Although I am still working on properly defining the functionality of the
MultiQueue, there are a few issues I see on the horizon which might render a
correct (and provably correct) structure difficult.

The first problem with this structure is the issues that it faces when one of
the threads is significantly faster than the others and, as a result, causes
the consumer or producer counter to wrap around to the point where another
thread is doing work. By having two threads operating on the same end of the
queue at once. The use of the MultiQueue essentially degenerates to the use of
the underlying concurrent queue. Theoretically, then, the MultiQueue does not
provide any correctness properties that are more powerful than the underlying
concurrent queue, and, thus, is strictly limited by it.

Another problem that the structure has which might compromise its correctness
is its use of counters for positioning producers and consumers. More
specifically, to check that a MultiQueue is non-empty, the current
implementation compares that the consumer counter is strictly less than the
producer counter; however, that raises the question of what should happen when
the producer counter (which uses the native integer size of the machine)
overflows and resets to zero. Practically, this will not likely happen for most
use cases as that would require that $2^{32}$ items be enqueued which is far
beyond most uses of concurrent queues, but there are some cases of long-running
applications or applications which which might reach that limit, and without a
guarantee of correctness for such a case, the MultiQueue is not a viable
option.


\subsection{Algorithmic Extensions}
There are several avenues I want to explore corresponding to the algorithms
used by the MultiQueue.

The first question I want to answer is: ``how large should the list of queues
be''. Initially, the concept was to have the size be the max of the number of
producer and consumer threads. This was in an attempt to have at most two
threads -- one consumer, one producer -- using a given queue at once, but, as I
have shown, it is possible for a single thread fast thread to wrap around the
queue and lead to multiple producers or consumers accessing a single queue. As
such, the question arises as to how to choose the correct number of queues for
the MultiQueue. One problem that I could take into consideration is that many
programs do not know -- in advance -- how many threads will be used, so having
a static size limit to the queues would prohibit use of the MultiQueue in these
cases. As such, it may be worth investigating whether it is possible to
dynamically resize the MultiQueue based on load.

In the other direction, there is the question of whether any guarantees can be
made about behavior if the size is kept static. Further, would it be possible
to guarantee that at most one producer and one consumer is accessing a given
internal queue at a time. If it were possible, then the internal queue could be
implemented using lock-less single producer single consumer queues, which could
lead to performance gains. In essence, this direction would involve moving the
complexity out of the internal queues and into the MultiQueue super structure.

Another direction which I will need to explore at least to some degree is the
question of what is the optimal concurrent queue algorithm to use in
conjunction with the MultiQueue. At the moment, my implementation uses a simple
2-lock structure provided by Micheal and Scott\cite{michael1996}. This is
because I my hypothesis is that the MultiQueue will work best when the
underlying concurrent queue algorithm is simple overall, and the given two lock
algorithm has a very small footprint. In contrast, lock free structures usually
have a good deal of overhead associated with them, the price of which is only
worth paying if there is a good deal of contention on between threads for the
same structure. If the MultiQueue is successful in negating the need for that
overhead by significantly reducing contention for a given queue, then using a
lock free internal queue would provide little benefit. Still, it is important
to gather empirical data to decide what internal queue to use. Regardless of
how I proceed, the answer to this question will be very useful to me.

In the trials I have done with the MultiQueue, I have found it is not uncommon
for a producer and a consumer to arrive an empty queue at the same time.
Elimination Back-off techniques have been used in concurrent stack
implementations to negate two opposite operations from different threads by
having the two threads communicate directly rather than through the stack.
Initially, I thought that because the domains of consumers and producers in
queues lie on separate ends of the structure, that the two would meet too
rarely for such techniques such to be useful, but my tests are suggesting it
may be worth implementing some sort of delegate to handle the case when a
consumer and producer meet in an empty queue. Such a delegate would require
some overhead for every operation on the queue, but could bypass the internal
queue structure altogether in some cases, possibly leading to performance
gains.

\subsection{Implementation Issues}
One of the main issue with implementing any sort of concurrent data structure
is that of how to store the data. If storing by value (the approach used by,
for example, C++ STL container classes), then there is a serious performance
penalty associated with copying large structures around for many operations. In
contrast, a more typical approach is to store data on the heap and pass
container objects by reference. This has the advantage that even large data
structure can be ``atomically'' changed with a CAS, but has the severe
limitation of needing to go through a memory allocator which often introduces
performance bottlenecks.

From the tests I have seen performed on concurrent queues, the use of
pass-by-reference greatly improves performance far above the use of
pass-by-value.\cite{suttertest}  As such, I implement all structures using pass
by reference, knowing that to get a more realistic view of how structures will
perform, I will need to implement an allocator which allows for a greater
degree of concurrency than the stock one. Herlihy et al. provide a non-blocking
memory management system which I plan to implement given the
time.\cite{herlihy2005}

\subsection{Testing}
There are two primary attributes I intend to test: relative speed and
scalability. For this, I am designing a flexible set of tests which test the
throughput of a queue -- the number of items that have been dequeued per
second.

The problem of measuring scalability is the more difficult of the two. The
metric I have decided to use is throughput: the number of items that have gone
through the queue in a given time period. The basic idea is to run some number
of threads as producers and some number as consumers, have them constantly try
to perform an operation on the queue, and then count the number of operations
that were completed in a given amount of time. With this scheme,
there are two primary variables which must be controlled.

Obviously, an ideal test would test a range of thread counts up to (and even
past) the number of processors on the system. This is to measure how adding
threads affects performance, and whether any noticeable patterns indicative of
common concurrency issues arise. If I only looked at this metric, I would have
to run $N$ tests for each algorithm, where $N$ is the number of available cores
on the testing machine. The issue with this metric, however, is that it does
not consider the effects of varying the relative number of consumers and
producers. As the tests on concurrent queues done by Sutter\cite{suttertest} show,
the points where throughput is maximized can vary widely depending on the
number of consumers and producers even when the number of threads is fixed.

The main challenge adding this second metric presents is that to get a full
picture, $N^2$ tests are required per algorithm. Further, for the sake of
granularity, it is desirable to test on a machine with many cores; however,
being an undergraduate student, it is difficult to procure access to one.
Running tests that may take several hours would make this task even more
difficult.

One approach to reduce the number of tests is to only test on the line where
$numberProducers+numberConsumers=N$ for the relative size tests, and again
where $numberProducers=numberConsumers$ for $threadCount=1\to N$. This amounts
to a total of $2N$ tests per algorithm. Although this would not give me a
complete picture, it is a reasonable compromise if I am unable to procure much
time on a large multi-core machine.

On the other hand, measuring relative speed against other algorithms is much
easier. As long as the testing configurations between algorithms stay the same,
I will not need special hardware. I will likely use the second set of tests
from the scalability tests for my final results, but the fact that I am able to
use my own machines to test relative speed gives me a good amount of leeway.

It is worth noting that Michael and Scott have posted their source code on
\url{ftp://ftp.cs.rochester.edu/pub/packages/sched\_conscious\_synch/concurrent\_queues},
so I will be able to adapt this code to use my test rig when testing.

\bibliographystyle{plain}
\bibliography{mid-report}
\end{document}
